\documentclass[CJKutf8,xcolor=pdftex,dvipsnames,table]{beamer}
\usepackage{hyperref}
\hypersetup{
  pdftitle={Signals and Systems},
  pdfauthor={Hong MingJian},
  pdfsubject={Introduction},
  pdfpagemode={FullScreen},
  colorlinks={true},
  linkcolor={blue},
}
\usepackage{CJKutf8}
\usepackage[export]{adjustbox}

\usetheme{Madrid}%{Warsaw}
\usecolortheme{default}

%gets rid of bottom navigation bars
\setbeamertemplate{footline}[page number]{}
%gets rid of navigation symbols
\setbeamertemplate{navigation symbols}{}

\begin{document}
\begin{CJK*}{UTF8}{song}

  \title{\CJKfamily{hei} 数字信号处理}
  \subtitle{\CJKfamily{hei} 第3章:系统}
  \author{\CJKfamily{hei} 洪明坚 \hspace{1mm} 桑军}
  \institute{\CJKfamily{hei} 重庆大学软件学院}
  \date{\today}

  \AtBeginSection[]
  {
    \begin{frame}
      \frametitle{Outline}
      \tableofcontents[currentsection]
    \end{frame}
  }

  \frame{\titlepage}
  \frame{\frametitle{目录}\tableofcontents}
  
  \section{系统}
  
  %% PAGE
  \begin{frame}
    \frametitle{定义}
    \begin{itemize}
    \item 新华字典:一种可以觉察的物理量或脉冲(如电压、电流、磁场强度等),通过它们能传达消息或信息。
    \item 维基百科:A signal is a function that conveys information about the behavior of a system or attributes of some phenomenon.
    \item 例子
        \begin{itemize}
        \item 声音、温度、电压/电流、光强等等
        \end{itemize}    
    \end{itemize}
  \end{frame}
    
  %% PAGE
  \begin{frame}
    \frametitle{Questions}
    \begin{itemize}
    \item Any questions?
    \end{itemize}
    \begin{center}
      \includegraphics[scale=.5]{question}
    \end{center}
  \end{frame}    
    
\end{CJK*}
\end{document}

%%% Local Variables: 
%%% mode: latex
%%% TeX-master: t
%%% End: 
