\documentclass[CJKutf8,dvipsnames,table]{beamer}
\usepackage{hyperref}
\hypersetup{
  pdfpagemode={FullScreen},
  colorlinks={true},
  linkcolor={blue},
}

%% https://tex.stackexchange.com/questions/47576/combining-ifxetex-and-ifluatex-with-the-logical-or-operation
\usepackage{iftex}
\newif\ifxetexorluatex % a new conditional starts as false
\ifnum 0\ifxetex 1\fi\ifluatex 1\fi>0
   \xetexorluatextrue
\fi
\usepackage{ifplatform}
\ifxetexorluatex
	\usepackage[slantfont,boldfont]{xeCJK}
	\ifwindows
		\setCJKmainfont{SimSun} % Windows默认中文字体:中易宋体
	\fi
	\ifmacosx
		\setCJKmainfont{STSong} % MacOS默认中文字体:华文宋体
	\fi
	\iflinux
		\setCJKmainfont{Noto Serif CJK SC} % Linux默认中文字体:思源宋体(By Adobe & Google)
	\fi
\else
	\usepackage{CJKutf8}
\fi

\usepackage[export]{adjustbox}
\usepackage{mathptmx} %pdfTeX error: pdflatex (file fmex9.pfb): cannot open Type 1 font file for reading
                                                 %https://forum.ubuntu.com.cn/viewtopic.php?t=269943
\usepackage{mathtools}
\usepackage[mathscr]{urwchancal}
\usepackage{amssymb}

\usetheme{Madrid}%{Warsaw}
\usecolortheme{default}

\setbeamertemplate{footline}[page number]{} %gets rid of bottom navigation bars
\setbeamertemplate{navigation symbols}{} %gets rid of navigation symbols

\title{数字信号处理}
\subtitle{第8讲:拉普拉斯变换}
\author{洪明坚}
\institute{重庆大学软件学院}
\date{\today}
  
\begin{document}
\ifxetexorluatex\else
\begin{CJK*}{UTF8}{song}  
\fi

  \AtBeginSection[]
  {
    \begin{frame}
      \frametitle{Outline}
      \tableofcontents[currentsection]
    \end{frame}
  }

  \frame{\titlepage}
  \frame{\frametitle{目录}\tableofcontents}
  
  \section{拉普拉斯变换}
  
  %% PAGE
  \begin{frame}
    \frametitle{回顾:连续时间LTI系统的系统函数}
    \begin{itemize}
    \item 假设$h(t)$是一个LTI系统的单位冲激响应,则该系统对输入$x(t)=e^{st}, s \in \mathbb{C}$的响应
    	\begin{align*}
 		y(t) & = x(t) * h(t) \\
		& = \int_R h(\tau)x(t - \tau )d\tau \\
		& = \int_R h(\tau )e^{s(t-\tau )}d\tau    \\
		& = e^{st} \int_R h(\tau)e^{-s\tau}d\tau \\
		& = H(s)e^{st}
    	\end{align*}   
		\begin{itemize}
		\item 称\[ H(s) = \int_R h(\tau)e^{-s\tau}d\tau \] 为LTI的系统函数(system function)
		\end{itemize}
    \end{itemize}
  \end{frame}  
    
  %% PAGE
  \begin{frame}
    \frametitle{定义}
    \begin{itemize}
    \item 对于一个信号$x(t)$,定义它的拉普拉斯变换(Laplace transform)如下
    \[
	    \mathscr{L}\{x(t)\} = X(s) \triangleq \int_R x(t)e^{-st}dt
    \]
    记为
    \[
    	x(t) \xleftrightarrow{\mathscr{L}} X(s)
    \]
    令$s=\sigma + j\omega$,可以写成
    \begin{align*}
    	X(s) = X(\sigma + j\omega) & = \int_R x(t)e^{-(\sigma + j\omega)t}dt \\
	                        & = \int_R [x(t)e^{-\sigma t}] e^{-j\omega t}dt \\
	                        & = \mathscr{F}\{ x(t) e^{-\sigma t} \}
    \end{align*}    
    当$\sigma=0$,即$s=j\omega$时,
    \[
    	\left. X(s) \right\vert_{s=j\omega} = \mathscr{F}\{x(t)\}
    \]
    拉普拉斯变换退化为傅立叶变换。
    \end{itemize}
  \end{frame}  
      
  %% PAGE
  \begin{frame}
    \frametitle{定义}
    \begin{center}
      \includegraphics[scale=.227]{segdsp-f32-1}
    \end{center}
  \end{frame}       
      
  %% PAGE
  \begin{frame}
    \frametitle{例子}
    \begin{itemize}
    \item 信号
    \[
    	x(t) = e^{-at}u(t)
    \]
    的傅立叶变换
    \[
    	X(j\omega) = \mathscr{F}\{e^{-at} u(t)\} = \int_{0}^{\infty} e^{-at} e^{-j\omega t} dt = \frac{1}{j\omega + a}, a > 0
    \]
    它的拉普拉斯变换
    \begin{align*}
    	X(s) = X(\sigma + j\omega) = \mathscr{F}\{e^{-(\sigma + a)t} u(t)\} & = \frac{1}{j\omega + (\sigma + a)}, \sigma+a>0 \\
	        & = \frac{1}{s+a}, \operatorname{Re}\{s\} > -a       
    \end{align*}
    \end{itemize}
  \end{frame} 
  
  %% PAGE
  \begin{frame}
    \frametitle{例子}
    \begin{itemize}
    \item 单位冲激函数
    \[
		\mathscr{L}\{\delta(t)\} = \int_R \delta(t)e^{-st} dt = 1
    \]
    \[
    	%https://en.wikipedia.org/wiki/Dirac_delta_function
		\mathscr{L}\{\delta'(t)\} = \int_R \delta'(t)e^{-st} dt = -\left. \frac{d}{dt}e^{-st} \right\vert_{t=0} = s
    \]
    \[
		\mathscr{L}\{\delta^{(k)}(t)\} = \int_R \delta^{(k)}(t)e^{-st} dt = (-1)^k\left. \frac{d^k}{dt^k}e^{-st} \right\vert_{t=0} = s^k
    \]
    \end{itemize}
  \end{frame} 
    
  %% PAGE
  \begin{frame}
    \frametitle{拉普拉斯变换存在与否}
    \begin{itemize}
    \item 不是所有信号都有拉普拉斯变换。前面的例子中,拉普拉斯变换存在的条件是$\operatorname{Re}\{s\} > -a$
		\begin{itemize}
		\item 当$a>0$时,傅立叶变换存在;
		\item 当$a \leq 0$时,傅立叶变换$X(j\omega)$不存在,但拉普拉斯变换$X(s)$存在。
		\end{itemize}
	\item 例子
    	\begin{itemize}
    	\item 信号
    	\[
    		x(t) = -e^{-at}u(-t)
    	\]
    	的拉普拉斯变换
    	\begin{align*}
    		X(s) = -\int_{-\infty}^{\infty} e^{-at} u(-t) e^{-st} dt & = -\int_{-\infty}^{0} e^{-(s+a)t} dt \\
	        	                                                     & = \frac{1}{s+a}       
    	\end{align*}
    	存在的条件是$\operatorname{Re}\{s+a\} < 0$,即$\operatorname{Re}\{s\} < -a$
    	\end{itemize}	
   	\end{itemize}
  \end{frame}  
       
  %% PAGE
  \begin{frame}
    \frametitle{拉普拉斯变换的收敛域}
    \begin{itemize}
    \item 把拉普拉斯变换存在时$s$的取值范围,称为收敛域(Region of convergence, ROC)
	\item 从前面的两个例子可以看出,不同的信号可能具有相同的拉普拉斯变换代数表达式,但它们的ROC却大不相同。
		\begin{itemize}
		\item 因此,在给出一个信号的拉普拉斯变换时,代数表达式和ROC应该同时给出。如 \\
	\begin{tabular}{ll}
	\raisebox{-.5\height}

		$
			e^{-at}u(t) \xleftrightarrow{\mathscr{L}} \frac{1}{s+a}, \operatorname{Re}\{s\} > -a
		$
&
    \includegraphics[valign=m,scale=.2]{ss-c-f9-1a}    \\
    \end{tabular} 

		和 \\
	\begin{tabular}{ll}
	\raisebox{-.5\height}

		$
			-e^{-at}u(-t) \xleftrightarrow{\mathscr{L}} \frac{1}{s+a}, \operatorname{Re}\{s\} < -a
		$
&
    \includegraphics[valign=m,scale=.2]{ss-c-f9-1b}    \\
    \end{tabular}  		

		\end{itemize}
   	\end{itemize}
  \end{frame} 
        
  %% PAGE
  \begin{frame}
    \frametitle{例子}
    \begin{itemize}
    \item 信号
    \[
    	x(t) = e^{-2t}u(t)+e^{-t}cos(3t)u(t) = [e^{-2t}+\frac{1}{2}e^{-(1-3j)t}+\frac{1}{2}e^{-(1+3j)t}]u(t)
    \]
    因为
    \[
    	e^{-2t}u(t) \xleftrightarrow{\mathscr{L}} \frac{1}{s+2}, \operatorname{Re}\{s\}>-2 
    \]
	\[
    	e^{-(1-3j)t}u(t) \xleftrightarrow{\mathscr{L}} \frac{1}{s+(1-3j)}, \operatorname{Re}\{s\}>-1 
    \]
	\[
    	e^{-(1+3j)t}u(t) \xleftrightarrow{\mathscr{L}} \frac{1}{s+(1+3j)}, \operatorname{Re}\{s\}>-1
    \]
    所以
    \begin{align*}
    	X(s) & = \frac{1}{s+2} + \frac{1}{2(s+(1-3j))} + \frac{1}{2(s+(1+3j))} \\
	         & = \frac{2s^2+5s+12}{(s^2+2s+10)(s+2)}, \operatorname{Re}\{s\}>-1
    \end{align*}
    \end{itemize}
  \end{frame}  
         
  %% PAGE
  \begin{frame}
    \frametitle{拉普拉斯变换的零点和极点}
    \begin{itemize}
    \item 上述例子的拉普拉斯变换都是有理的,即可以表示为复变量$s$的两个多项式之比
    \[
    	X(s) = \frac{N(s)}{D(s)}
    \]
    因为多项式可以用它的根来表示,所以
	\[
		X(s) = M\frac{\prod_{i=1}^{R}(s-\beta_i)}{\prod_{j=1}^{P}(s-\alpha_j)}
	\]   
    	\begin{itemize}
    	\item 因为$X(\alpha_j)=\infty$,称$\alpha_j$为极点(pole)
    	\item 因为$X(\beta_i)=0$,称$\beta_i$为零点(zero)
		\item 如果$R > P$,$X(s)$在$\infty$处有一个$(R-P)$阶极点
		\item 如果$R < P$,$X(s)$在$\infty$处有一个$(P-R)$阶零点
    	\end{itemize}	 
    \end{itemize}
  \end{frame}  
  
  %% PAGE
  \begin{frame}
    \frametitle{零点-极点图}
    \begin{itemize}
	\item 用$s$平面上的零点($\bigcirc$)和极点($\times$)来表示$X(s)$,称为$X(s)$的\textbf{零点-极点图(pole-zero plot)}
		\begin{itemize}
		\item 除了一个常数因子外,一个有理拉普拉斯变换可以用零点-极点图和ROC完全表征	
		\end{itemize}
	\item 例子
	\begin{align*}
	   	X(s) & = \frac{1}{s+2} + \frac{1}{2(s+(1-3j))} + \frac{1}{2(s+(1+3j))} = \frac{2s^2+5s+12}{(s^2+2s+10)(s+2)} \\
		     & = \frac{(s+\frac{5+j\sqrt{71}}{4})(s+\frac{5-j\sqrt{71}}{4})}{(s+2)(s+(1-3j))(s+(1+3j))}, \operatorname{Re}\{s\}>-1
	\end{align*}
	的零点-极点图和ROC。注意$X(s)$在$\infty$处有一个零点。
		\begin{center}
    	\includegraphics[scale=.28]{ss-c-f9-2b}
    	\end{center}
    \end{itemize}
  \end{frame}  
      
  %% PAGE
  \begin{frame}
    \frametitle{例子}
    \begin{itemize}
	\item 信号
	\[
		x(t) = \delta(t) - \frac{4}{3}e^{-t}u(t)+\frac{1}{3}e^{2t}u(t)
	\]
	因为
	\[
		\mathscr{L}\{\delta(t)\} = \int_R \delta(t)e^{-st} dt = 1
	\]
	所以
	\[
	   	X(s)  = 1 - \frac{4}{3}\frac{1}{s+1} + \frac{1}{3}\frac{1}{s-2} = \frac{(s-1)^2}{(s+1)(s-2)}, \operatorname{Re}\{s\} > 2
	\]
		\begin{center}
    	\includegraphics[scale=.47]{ss-c-f9-3}
    	\end{center}
    \end{itemize}
  \end{frame}  
                  
  %% PAGE
  \begin{frame}
    \frametitle{Questions}
    \begin{itemize}
    \item Any questions?
    \end{itemize}
    \begin{center}
      \includegraphics[scale=.5]{question}
    \end{center}
  \end{frame} 
  
  \section{收敛域的性质} 
  
  %% PAGE
  \begin{frame}
    \frametitle{性质}
    \begin{itemize}
    \item 1、$X(s)$的ROC在$s$平面内由平行于$j\omega$轴的带状区域组成。
    	\begin{itemize}
		\item 因为
    		\begin{align*}
    			X(s) = X(\sigma + j\omega) & = \int_R x(t)e^{-(\sigma + j\omega)t}dt \\
	                                       & = \int_R [x(t)e^{-\sigma t}] e^{-j\omega t}dt \\
	                                       & = \mathscr{F}\{ x(t) e^{-\sigma t} \}
    		\end{align*}
		而$\mathscr{F}\{ x(t) e^{-\sigma t} \}$存在的条件
		\[
			\int_R |x(t)|e^{-\sigma t}dt < \infty
		\]
		只与$\operatorname{Re}\{s\}=\sigma$有关,与$\operatorname{Im}\{s\}=\omega$无关。
		\end{itemize}
    \end{itemize}
  \end{frame} 
    
  %% PAGE
  \begin{frame}
    \frametitle{性质}
    \begin{itemize}
    \item 2、对有理拉普拉斯变换来说,ROC内不包含任何极点。
    \item 3、如果$x(t)$是有限持续期,且是绝对可积的,那么它的ROC是整个$s$平面
    	\begin{itemize}
		\item 因为
    		\[
    			\int_{T_1}^{T_2} |x(t)| dt < \infty \Rightarrow \{\operatorname{Re}\{s\}=\sigma = 0\} \subset ROC
    		\]
		以及
		\begin{align*}
			\int_{T_1}^{T_2} |x(t)|e^{-\sigma t}dt & < e^{-\sigma T_1}\int_{T_1}^{T_2} |x(t)| dt \Rightarrow \{\operatorname{Re}\{s\}=\sigma > 0\} \subset ROC \\
			\int_{T_1}^{T_2} |x(t)|e^{-\sigma t}dt & < e^{-\sigma T_2}\int_{T_1}^{T_2} |x(t)| dt \Rightarrow \{\operatorname{Re}\{s\}=\sigma < 0\} \subset ROC
		\end{align*}
		所以,ROC是整个$s$平面。
		
		\item 已知
	\begin{tabular}{ll}
	\raisebox{-.5\height}

    \begin{math}
x(t) = 
\left\{
    \begin {aligned}
         & e^{-at} \quad & 0 < t < T \\
         & 0 \quad & otherwise                 
    \end{aligned}
\right.
	\end{math}
&
     $\xleftrightarrow{\mathscr{L}} \frac{1}{s+a}[1-e^{-(s+a)T}]$ 
    \end{tabular}
     ,问:$s=-a$是不是极点?
		\end{itemize}
    \end{itemize}
  \end{frame} 
      
  %% PAGE
  \begin{frame}
    \frametitle{性质}
    \begin{itemize}
    \item 4、如果$x(t)$是右边(right-sided)信号,且$\operatorname{Re}\{s\}=\sigma_0$这条线位于ROC内,那么$\operatorname{Re}\{s\}>\sigma_0$的全部$s$值一定在ROC内。   
		\begin{tabular}{ll}
		\includegraphics[scale=.45]{ss-c-f9-6}
		&
     	\includegraphics[scale=.3]{ss-c-f9-7}
    	\end{tabular} 
    	\[  
     	\int_{T_1}^{\infty} |x(t)|e^{-\sigma_1 t}dt  < \int_{T_1}^{\infty} |x(t)| e^{-\sigma_0 T_1} dt, \sigma_1 > \sigma_0
    	\]
    \item 5、如果$x(t)$是左边(left-sided)信号,且$\operatorname{Re}\{s\}=\sigma_0$这条线位于ROC内,那么$\operatorname{Re}\{s\}<\sigma_0$的全部$s$值一定在ROC内。   
    \end{itemize}
  \end{frame} 

  %% PAGE
  \begin{frame}
    \frametitle{性质}
    \begin{itemize}
    \item 6、如果$x(t)$是双边(two-sided)信号,且$\operatorname{Re}\{s\}=\sigma_0$这条线位于ROC内,那么ROC一定由$s$平面的一条带状区域组成,且直线$\operatorname{Re}\{s\}=\sigma_0$位于带中。   
    \begin{itemize}
    \item 例子
    \[
    	x(t) = e^{-b|t|}=e^{-bt}u(t)+e^{bt}u(-t)
    \]
    \[
    	e^{-bt}u(t) \xleftrightarrow{\mathscr{L}} \frac{1}{s+b}, \operatorname{Re}\{s\} > -b
    \]
    \[
    	e^{bt}u(-t) \xleftrightarrow{\mathscr{L}} \frac{-1}{s-b}, \operatorname{Re}\{s\} < b
    \]
    尽管两个单独项都有拉普拉斯变换,但是当$b \leq 0$时,$x(t)$没有拉普拉斯变换。当$b>0$时,有
    \[
    	e^{-b|t|} \xleftrightarrow{\mathscr{L}} \frac{1}{s+b} - \frac{1}{s-b}=\frac{-2b}{(s-b)(s+b)}, -b < \operatorname{Re}\{s\} < b
    \]
    
    \end{itemize}
		\begin{center}
		\begin{tabular}{ll}
		\includegraphics[scale=.2]{ss-c-f9-11a}
		&
		%\includegraphics[scale=.23]{ss-c-f9-11b}
		%&
     	\includegraphics[scale=.2]{ss-c-f9-12}
    	\end{tabular} 
		\end{center}
    \end{itemize}
  \end{frame}      
         
  %% PAGE
  \begin{frame}
    \frametitle{性质}
    \begin{itemize}
    \item 7、如果$x(t)$的拉普拉斯变换$X(s)$是有理的,那么ROC是被极点所界定的或延伸到$\infty$,而且在ROC内不包含任何极点。   
    \item 8、如果$x(t)$的拉普拉斯变换$X(s)$是有理的,那么
    	\begin{itemize}
    	\item 若$x(t)$是右边信号,则ROC在s平面位于最右边极点的右边;
    	\item 若$x(t)$是左边信号,则ROC在s平面位于最左边极点的左边。
    	\end{itemize}
	\item 例子
		\begin{itemize}
		\item 某一拉普拉斯变换的代数表达式为
		\[
			X(s) = \frac{1}{(s+1)(s+2)}
		\]
		它的极点分布,以及三种可能的ROC
		\end{itemize}
		\begin{center}
    	\includegraphics[scale=.42]{ss-c-f9-13}
    	\end{center}
    \end{itemize}
  \end{frame}
         
  %% PAGE
  \begin{frame}
    \frametitle{Questions}
    \begin{itemize}
    \item Any questions?
    \end{itemize}
    \begin{center}
      \includegraphics[scale=.5]{question}
    \end{center}
  \end{frame} 
  
  \section{拉普拉斯逆变换}
  
  %% PAGE
  \begin{frame}
    \frametitle{拉普拉斯逆变换}
    \begin{itemize}
    \item 根据拉普拉斯变换的定义
	\[
    	X(s) = X(\sigma + j\omega) = \mathscr{F}\{ x(t) e^{-\sigma t} \}
	         = \int_R x(t)e^{-(\sigma + j\omega)t}dt 
	\]
	\item 利用傅立叶逆变换
	\[
    	x(t) e^{-\sigma t} = \mathscr{F}^{-1}\{ X(\sigma + j\omega) \}   
	                       = \frac{1}{2\pi}\int_R X(\sigma + j\omega)e^{j\omega t}d\omega 
	\]
	两边同乘以$e^{\sigma t}$得到
	\[
		x(t) = \frac{1}{2\pi}\int_R X(\sigma + j\omega)e^{(\sigma+j\omega) t}d\omega 
	\]
	因为$s=\sigma+j\omega$,固定$\sigma$,有$ds=jd\omega$
	\[
		x(t) = \frac{1}{2\pi j}\int_{\sigma-j\omega}^{\sigma+j\omega} X(s)e^{st}ds 	
	\]
	该式对于任意一条在ROC中的直线$\operatorname{Re}\{s\}=\sigma$都成立。
    \end{itemize}
  \end{frame}   

  %% PAGE
  \begin{frame}
    \frametitle{拉普拉斯逆变换}
    \begin{itemize}
    \item 用上述的定义求拉普拉斯逆变换涉及到复变函数中的围线积分(contour integration),比较麻烦。对于有理变换,有更加简便的方法。
	\item 有理拉普拉斯变换可以展开成部分分式之和
	\[
		X(s) = \sum_{i=1}^{r} \sum_{k_i=1}^{N_i} \frac{A_{i,k_i}}{(s+a_i)^{k_i}}
	\]
	利用已知的变换
	\begin{align*}
		\frac{t^{n-1}}{(n-1)!}e^{-at}u(t) & \xleftrightarrow{\mathscr{L}} \frac{1}{(s+a)^n}, \operatorname{Re}\{s\} > -a \\
		-\frac{t^{n-1}}{(n-1)!}e^{-at}u(-t) & \xleftrightarrow{\mathscr{L}} \frac{1}{(s+a)^n}, \operatorname{Re}\{s\} < -a
	\end{align*}
    \end{itemize}
  \end{frame}   
    
  %% PAGE
  \begin{frame}
    \frametitle{例子}
    \begin{itemize}
    \item 考虑如下拉普拉斯变换
	\[
	   	X(s)  = \frac{1}{(s+1)(s+2)^2} = \frac{1}{s+1} - \frac{1}{s+2} - \frac{1}{(s+2)^2}
	\]
	有两个极点$s=-1$和$s=-2$。
		\begin{itemize}
		\item 1、如果ROC是$\operatorname{Re}\{s\} > -1$,因为
		\begin{align*}
			e^{-t}u(t) & \xleftrightarrow{\mathscr{L}} \frac{1}{s+1}, \operatorname{Re}\{s\} > -1 \\
			e^{-2t}u(t) & \xleftrightarrow{\mathscr{L}} \frac{1}{s+2}, \operatorname{Re}\{s\} > -2 \\
			te^{-2t}u(t) & \xleftrightarrow{\mathscr{L}} \frac{1}{(s+2)^2}, \operatorname{Re}\{s\} > -2 
		\end{align*}
		所以
		\[
			x(t) = e^{-t}u(t) - e^{-2t}u(t) - te^{-2t}u(t)
		\]
		\end{itemize}
    \end{itemize}
  \end{frame}  
      
  %% PAGE
  \begin{frame}
    \frametitle{例子}
    \begin{itemize}
    \item 考虑如下拉普拉斯变换
	\[
	   	X(s)  = \frac{1}{(s+1)(s+2)^2} = \frac{1}{s+1} - \frac{1}{s+2} - \frac{1}{(s+2)^2}
	\]
	有两个极点$s=-1$和$s=-2$。
		\begin{itemize}
		\item 2、如果ROC是$\operatorname{Re}\{s\} < -2$,因为
		\begin{align*}
			-e^{-t}u(-t) & \xleftrightarrow{\mathscr{L}} \frac{1}{s+1}, \operatorname{Re}\{s\} < -1 \\
			-e^{-2t}u(-t) & \xleftrightarrow{\mathscr{L}} \frac{1}{s+2}, \operatorname{Re}\{s\} < -2 \\
			-te^{-2t}u(-t) & \xleftrightarrow{\mathscr{L}} \frac{1}{(s+2)^2}, \operatorname{Re}\{s\} < -2 
		\end{align*}
		所以
		\[
			x(t) = -e^{-t}u(-t) + e^{-2t}u(-t) + te^{-2t}u(-t)
		\]
		\end{itemize}
    \end{itemize}
  \end{frame}  
        
  %% PAGE
  \begin{frame}
    \frametitle{例子}
    \begin{itemize}
    \item 考虑如下拉普拉斯变换
	\[
	   	X(s)  = \frac{1}{(s+1)(s+2)^2} = \frac{1}{s+1} - \frac{1}{s+2} - \frac{1}{(s+2)^2}
	\]
	有两个极点$s=-1$和$s=-2$。
		\begin{itemize}
		\item 3、如果ROC是$-2 < \operatorname{Re}\{s\} < -1$,因为
		\begin{align*}
			-e^{-t}u(-t) & \xleftrightarrow{\mathscr{L}} \frac{1}{s+1}, \operatorname{Re}\{s\} < -1 \\
			e^{-2t}u(t) & \xleftrightarrow{\mathscr{L}} \frac{1}{s+2}, \operatorname{Re}\{s\} > -2 \\
			te^{-2t}u(t) & \xleftrightarrow{\mathscr{L}} \frac{1}{(s+2)^2}, \operatorname{Re}\{s\} > -2 
		\end{align*}
		所以
		\[
			x(t) = -e^{-t}u(-t) - e^{-2t}u(t) - te^{-2t}u(t)
		\]
		\end{itemize}
    \end{itemize}
  \end{frame}  
    
  %% PAGE
  \begin{frame}
    \frametitle{基本函数的拉普拉斯变换}
    \begin{center}
      \includegraphics[scale=.37]{ss-c-t9-2}
    \end{center}
  \end{frame}   
            
  %% PAGE
  \begin{frame}
    \frametitle{Questions}
    \begin{itemize}
    \item Any questions?
    \end{itemize}
    \begin{center}
      \includegraphics[scale=.5]{question}
    \end{center}
  \end{frame}
  
  \section{用零极图分析频率响应特性}
  
  %% PAGE
  \begin{frame}
    \frametitle{零点向量和极点向量}
    \begin{itemize}
	\item 如果
	\[
	X(s)=s-a
	\]
	$a$是一个零点,对于$s$平面中任意一点$s_1$,称$s_1-a$为零点向量
		\begin{itemize}
		\item $X(s_1)$的模就是向量的长度,相位就是该向量与实轴$\operatorname{Re}$的夹角
		\end{itemize}
	\item 如果
	\[
	X(s)=\frac{1}{s-a}
	\]
	$a$是一个极点,对于$s$平面中任意一点$s_1$,称$s_1-a$为极点向量
		\begin{itemize}
		\item $X(s_1)$的模就是向量的长度的倒数,相位就是该向量与实轴$\operatorname{Re}$的夹角的负值
		\end{itemize}

    \end{itemize}
    \begin{center}
      \includegraphics[scale=.5]{ss-c-f9-15}
    \end{center}
  \end{frame}
  
  %% PAGE
  \begin{frame}
    \frametitle{用零极图分析频率响应特性}
    \begin{itemize}
    \item 对于一般的有理拉普拉斯变换
	\[
		X(s) = M\frac{\prod_{i=1}^{R}(s-\beta_i)}{\prod_{j=1}^{P}(s-\alpha_j)}
	\]
	$X(s_1)$的模就是各零点向量长度之积的M倍除以各极点向量长度之积,相位就是零点相位之和减去极点的相位之和
    \end{itemize}
  \end{frame}    
  
  %% PAGE
  \begin{frame}
    \frametitle{用零极图分析频率响应特性}
    \begin{itemize}
    \item 例子1   
\[
X(s)=\frac{1}{s+1/\tau}, \operatorname{Re}\{s\} > -1/\tau
\]
它的傅立叶变换
\[
X(j\omega)=\frac{1}{j\omega+1/\tau}
\]
它的零点-极点图($\tau=2$)
    \begin{center}
      \includegraphics[scale=.5]{ss-c-f9-16}
    \end{center}
    和频率响应特性
    \begin{center}
      \includegraphics[scale=.27]{ss-c-f9-18a}
      \includegraphics[scale=.27]{ss-c-f9-18b}
    \end{center}

    \end{itemize}
     
  \end{frame}     
  
  %% PAGE
  \begin{frame}
    \frametitle{用零极图分析频率响应特性}
    \begin{itemize}
    \item 例子2   
\[
X(s)=\frac{s-a}{s+a}, \operatorname{Re}\{s\} > -a
\]
它的零点-极点图和频率响应特性
    \begin{center}
      \includegraphics[scale=.22]{ss-c-f9-21}
    \end{center}
    	\begin{itemize}
    	\item 
    	\[
    	|H(j\omega)| = 1
    	\]
    	\[    
    	arg\{ H(j\omega) \}= \theta_1 - \theta_2 = \pi - 2\theta_2 = \pi - 2arctan(\frac{\omega}{a})
    	\]
    	\end{itemize}
	是一个全通系统(all-pass system)。
    \end{itemize}     
  \end{frame}     
    
  %% PAGE
  \begin{frame}
    \frametitle{Questions}
    \begin{itemize}
    \item Any questions?
    \end{itemize}
    \begin{center}
      \includegraphics[scale=.5]{question}
    \end{center}
  \end{frame}  
  
  \section{拉普拉斯变换的性质}
  
  %% PAGE
  \begin{frame}
    \frametitle{性质}
    \begin{itemize}
    \item 1、线性
    若
    \[
    x_1(t) \xleftrightarrow{\mathscr{L}} X_1(s), ROC=R_1
    \]
    \[
    x_2(t) \xleftrightarrow{\mathscr{L}} X_2(s), ROC=R_2
    \]
    则
    \[
    ax_1(t)+bx_2(t) \xleftrightarrow{\mathscr{L}} aX_1(s)+bX_2(s), R_1 \cap R_2 \subset ROC 
    \]
    	\begin{itemize}
    	\item 如果$R_1 \cap R_2=\emptyset$,则拉普拉斯变换不存在。
   		\item 注意:ROC可能比$R_1 \cap R_2$大。
			\begin{itemize}
			\item 如果$x_1(t)=x_2(t)$,且$a=-b$,则$x(t)=0$,因此$X(s)=0$,ROC是整个s平面。
			\end{itemize}
    	\end{itemize}
    \end{itemize}
  \end{frame} 
    
  %% PAGE
  \begin{frame}
    \frametitle{性质}
    \begin{itemize}
    \item 2、时域平移 \\
    若
    \[
    x(t) \xleftrightarrow{\mathscr{L}} X(s), ROC=R
    \]
    则
    \[
    x(t-t_0) \xleftrightarrow{\mathscr{L}} e^{-st_0}X(s), ROC=R
    \]
    \item 3、$s$域平移 
    \[
    e^{s_0 t}x(t) \xleftrightarrow{\mathscr{L}} X(s-s_0), ROC=R+\operatorname{Re}\{s_0\}
    \]
    \begin{center}
      \includegraphics[scale=.5]{ss-c-f9-23}
    \end{center}
    \end{itemize}
  \end{frame}
      
  %% PAGE
  \begin{frame}
    \frametitle{性质}
    \begin{itemize}
    \item 4、卷积 \\
    \[
    x_1(t) \ast x_2(t) \xleftrightarrow{\mathscr{L}} X_1(s)X_2(s), R_1 \cap R_2 \subset ROC 
    \]
    	\begin{itemize}
   		\item 注意:ROC可能比$R_1 \cap R_2$大。
			\begin{itemize}
			\item 如果
			\[
			X_1(s)=\frac{s+1}{s+2}, \operatorname{Re}\{s\} > -2
			\]
			\[
			X_1(s)=\frac{s+2}{s+1}, \operatorname{Re}\{s\} > -1
			\]
			则$X_1(s)X_2(s)=1$,ROC是整个s平面。
			\end{itemize}
    	\end{itemize}
    \end{itemize}
  \end{frame}
   
  %% PAGE
  \begin{frame}
    \frametitle{性质}
    \begin{itemize}
    \item 5、时域微分 \\
    \[
    \frac{dx(t)}{dt} \xleftrightarrow{\mathscr{L}} sX(s), R \subset ROC 
    \]
    	\begin{itemize}
		\item 方法1:对下式求$t$的导数
		\[
				x(t) = \frac{1}{2\pi j}\int_{\sigma-j\omega}^{\sigma+j\omega} X(s)e^{st}ds 	
		\]
		可得
		\[
				\frac{d}{dt}x(t) = \frac{1}{2\pi j}\int_{\sigma-j\omega}^{\sigma+j\omega} [sX(s)]e^{st}ds 	
		\]
		\item 方法2:利用卷积性质和$\delta(t)$的性质
		\[
			x'(t) = x(t) \ast \delta'(t) %https://math.stackexchange.com/questions/1912108/convolution-between-the-derivative-dirac-delta-function-and-other-function
		\]
		有
		\[
		\mathscr{L}\{x'(t)\} = \mathscr{L}\{x(t) \ast \delta'(t) \} = \mathscr{L}\{x(t)\} \mathscr{L}\{\delta'(t)\} = \mathscr{L}\{x(t)\}s = sX(s)
		\]
		\end{itemize}
    \end{itemize}
  \end{frame}

  %% PAGE
  \begin{frame}
    \frametitle{性质}
    \begin{itemize}
    \item 6、$s$域微分
    \[
    -tx(t) \xleftrightarrow{\mathscr{L}} \frac{dX(s)}{ds}, ROC=R
    \]
    	\begin{itemize}
    	\item 例子
    	\[
    		e^{-at}u(t) \xleftrightarrow{\mathscr{L}} \frac{1}{s+a}, \operatorname{Re}\{s\} > -a     
    	\]
		利用微分性质
		\[
			te^{-at}u(t) \xleftrightarrow{\mathscr{L}} -\frac{d}{ds}[\frac{1}{s+a}]=\frac{1}{(s+a)^2}, \operatorname{Re}\{s\} > -a 
		\]
		重复利用可得
		\[
			\frac{t^{n-1}}{(n-1)!}e^{-at}u(t) \xleftrightarrow{\mathscr{L}} \frac{1}{(s+a)^n}, \operatorname{Re}\{s\} > -a 		
		\]    
    \end{itemize}
    \end{itemize}
  \end{frame}

  %% PAGE
  \begin{frame}
    \frametitle{性质汇总}
    \begin{center}
      \includegraphics[scale=.4]{ss-c-t9-1}
    \end{center}
  \end{frame}              
          
  %% PAGE
  \begin{frame}
    \frametitle{Questions}
    \begin{itemize}
    \item Any questions?
    \end{itemize}
    \begin{center}
      \includegraphics[scale=.5]{question}
    \end{center}
  \end{frame}   
  
  \section{用拉普拉斯变换分析LTI系统}
  
  %% PAGE
  \begin{frame}
    \frametitle{LTI系统函数}
    \begin{itemize}
    \item 如果$h(t)$是LTI系统的冲激响应
    \[
    	y(t) = x(t) \ast h(t)
    \]
    根据拉普拉斯变换的卷积性质
    \[
    	Y(s) = X(s)H(s)
    \]
	称
	\[
	H(s) = \int_R h(t)e^{-st}dt
	\]
	为系统函数或转移函数(transfer function)
    \end{itemize}
  \end{frame}   
  
  %% PAGE
  \begin{frame}
    \frametitle{因果性}
    \begin{itemize}
    \item 对于因果的LTI系统,有$h(t)=0, t < 0$,
	%\[
	%	H(s) = \int_R h(t)e^{-st}dt = \int_{0}^{\infty} h(t) e^{-(\sigma+j\omega)t}dt = \int_{0}^{\infty} h(t) e^{-\sigma t} e^{-j\omega t}dt
	%\]
	则它的ROC是某个右半平面
		\begin{itemize}
		\item 例子
		\[
			h(t) = e^{-t}u(t) \xleftrightarrow{\mathscr{L}} H(s) = \frac{1}{s+1}, \operatorname{Re}\{s\} > -1
		\]
		\end{itemize}
	\item 反过来却不成立
		\begin{itemize}
		\item 例子
		\[
			h(t) = e^{-(t+1)}u(t+1) \xleftrightarrow{\mathscr{L}} H(s) = \frac{e^s}{s+1}, \operatorname{Re}\{s\} > -1
		\]
		$h(t) \neq 0, -1 < t < 0$,所以它不是因果的。
		\end{itemize}
	
    \end{itemize}
  \end{frame}   
    
  %% PAGE
  \begin{frame}
    \frametitle{稳定性}
    \begin{itemize}
    \item 对于稳定的LTI系统,有
    \[
    	\int_R |h(t)|dt < \infty
    \]
    此时,单位冲激响应的傅立叶变换(即频率响应)$H(j\omega)$存在。所以,
    \item 当且仅当系统函数$H(s)$的ROC包含$s$平面的虚轴$j\omega$,即$\operatorname{Re}\{s\}=0$时,LTI系统是稳定的。
	\item 综合起来:一个因果的LTI系统是稳定的,当且仅当$H(s)$最右边的极点必须位于$j\omega$轴的左边,即所有极点的实部必须是负数。
	
    \end{itemize}
  \end{frame}   
      
  %% PAGE
  \begin{frame}
    \frametitle{由线性常系数微分方程表征的LTI系统}
    \begin{itemize}
    \item 一个连续时间LTI系统可以用线性常系数微分方程表示
    \[
    	\sum_{k=0}^{N} a_k \frac{d^k y(t)}{dt^k} = \sum_{k=0}^{M} b_k \frac{d^k x(t)}{dt^k}
    \]
    两边进行拉普拉斯变换,反复应用线性和微分性质,可得
    \[
    	(\sum_{k=0}^{N} a_k s^k)Y(s)  = (\sum_{k=0}^{M} b_k s^k ) X(s)
    \]
    因此
    \[
    	H(s) = \frac{Y(s)}{X(s)} = \frac{\sum_{k=0}^{M} b_k s^k}{\sum_{k=0}^{N} a_k s^k}    
    \]
    即,一个由线性常系数微分方程表征的LTI系统,且系统函数总是有理的。
    \end{itemize}
  \end{frame}    
  
  %% PAGE
  \begin{frame}
    \frametitle{例子}
    \begin{itemize}
    \item 对于下面的系统
    \[
    	\frac{dy(t)}{dt} + 3y(t) = x(t)
    \]
    两边进行拉普拉斯变换,应用线性和微分性质,可得
    \[
    	sY(s) + 3Y(s) = X(s)
    \]
    因此
    \[
    	H(s) = \frac{Y(s)}{X(s)} = \frac{1}{s+3}    
    \]
    	\begin{itemize}
		\item 如果ROC是$\operatorname{Re}\{s\} > -3$,则$h(t) = e^{-3t}u(t)$
		\item 如果ROC是$\operatorname{Re}\{s\} < -3$,则$h(t) = -e^{-3t}u(-t)$	
		\end{itemize}
		
	\item 可以证明:对于一个有理的LTI系统,系统是因果的等价于ROC位于最右边极点的右边。
    \end{itemize}
  \end{frame}
    
            
  %% PAGE
  \begin{frame}
    \frametitle{Questions}
    \begin{itemize}
    \item Any questions?
    \end{itemize}
    \begin{center}
      \includegraphics[scale=.5]{question}
    \end{center}
  \end{frame}    
  
\ifxetexorluatex\else
\end{CJK*}  
\fi  
\end{document}

%%% Local Variables: 
%%% mode: latex
%%% TeX-master: t
%%% End: 
